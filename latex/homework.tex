\documentclass{article}

\usepackage{fancyhdr}
\usepackage{extramarks}
\usepackage{amsmath}
\usepackage{amsthm}
\usepackage{amsfonts}
\usepackage{tikz}
\usepackage[plain]{algorithm}
\usepackage{algpseudocode}

\usetikzlibrary{automata,positioning}


\usepackage{listings}
\usepackage{xcolor}



\definecolor{mylightgray}{rgb}{0.95,0.95,0.95}

\lstset{%
  backgroundcolor=\color{mylightgray},
  basicstyle=\ttfamily\footnotesize,
  keywordstyle=\color{blue},
  commentstyle=\color{gray},
  stringstyle=\color{green},
  numbers=left,
  numberstyle=\tiny\color{gray},
  stepnumber=1,
  numbersep=5pt,
  breaklines=true,
  breakatwhitespace=false,
  showspaces=false,
  showstringspaces=false,
  showtabs=false,
  frame=single,
  tabsize=4,
  captionpos=b,
  language=Matlab
}




%
% Basic Document Settings
%

\topmargin=-0.45in
\evensidemargin=0in
\oddsidemargin=0in
\textwidth=6.5in
\textheight=9.0in
\headsep=0.25in

\linespread{1.1}

\pagestyle{fancy}
\lhead{\hmwkAuthorName}
\chead{\hmwkClass\ : \hmwkTitle}
\rhead{\firstxmark}
\lfoot{\lastxmark}
\cfoot{\thepage}

\renewcommand\headrulewidth{0.4pt}
\renewcommand\footrulewidth{0.4pt}

\setlength\parindent{0pt}

%
% Create Problem Sections
%

\newcommand{\enterProblemHeader}[1]{
    \nobreak\extramarks{}{Problem \arabic{#1} continued on next page\ldots}\nobreak{}
    \nobreak\extramarks{Problem \arabic{#1} (continued)}{Problem \arabic{#1} continued on next page\ldots}\nobreak{}
}

\newcommand{\exitProblemHeader}[1]{
    \nobreak\extramarks{Problem \arabic{#1} (continued)}{Problem \arabic{#1} continued on next page\ldots}\nobreak{}
    \stepcounter{#1}
    \nobreak\extramarks{Problem \arabic{#1}}{}\nobreak{}
}

\setcounter{secnumdepth}{0}
\newcounter{partCounter}
\newcounter{homeworkProblemCounter}
\setcounter{homeworkProblemCounter}{1}
\nobreak\extramarks{Problem \arabic{homeworkProblemCounter}}{}\nobreak{}

%
% Homework Problem Environment
%
% This environment takes an optional argument. When given, it will adjust the
% problem counter. This is useful for when the problems given for your
% assignment aren't sequential. See the last 3 problems of this template for an
% example.
%
\newenvironment{homeworkProblem}[1][-1]{
    \ifnum#1>0
        \setcounter{homeworkProblemCounter}{#1}
    \fi
    \section{Problem \arabic{homeworkProblemCounter}}
    \setcounter{partCounter}{1}
    \enterProblemHeader{homeworkProblemCounter}
}{
    \exitProblemHeader{homeworkProblemCounter}
}

%
% Homework Details
%   - Title
%   - Due date
%   - Class
%   - Section/Time
%   - Instructor
%   - Author
%

\newcommand{\hmwkTitle}{Final Project}
\newcommand{\hmwkClass}{Introduction To Image and Video Processing}
\newcommand{\hmwkClassInstructor}{Professor Nasser Nasrabadi}
\newcommand{\hmwkAuthorName}{\textbf{Komel Merchant}}

%
% Title Page
%

\title{
    \vspace{2in}
    \textmd{\textbf{\hmwkClass:\ \hmwkTitle}}\\
    \vspace{0.1in}\large{\textit{\hmwkClassInstructor\ \hmwkClassTime}}
    \vspace{3in}
}

\author{\hmwkAuthorName}
\date{}

\renewcommand{\part}[1]{\textbf{\large Part \Alph{partCounter}}\stepcounter{partCounter}\\}

%
% Various Helper Commands
%

% Useful for algorithms
\newcommand{\alg}[1]{\textsc{\bfseries \footnotesize #1}}

% For derivatives
\newcommand{\deriv}[1]{\frac{\mathrm{d}}{\mathrm{d}x} (#1)}

% For partial derivatives
\newcommand{\pderiv}[2]{\frac{\partial}{\partial #1} (#2)}

% Integral dx
\newcommand{\dx}{\mathrm{d}x}

% Alias for the Solution section header
\newcommand{\solution}{\textbf{\large Solution}}

% Probability commands: Expectation, Variance, Covariance, Bias
\newcommand{\E}{\mathrm{E}}
\newcommand{\Var}{\mathrm{Var}}
\newcommand{\Cov}{\mathrm{Cov}}
\newcommand{\Bias}{\mathrm{Bias}}



\begin{document}

\maketitle

\pagebreak

\begin{homeworkProblem}

	\textbf{Part 1 [question(s): 1, 2]}
	\\
	Using the standard hierarchical reconstruction methods and using zero-images, I generated plots for all three cases in question 1 in Figure \ref{fig:p1}. Entropy values for each sub-band can be found in plot \ref{fig:p2}.
\end{homeworkProblem}


\begin{homeworkProblem}

	\textbf{Part 2 [question(s): 3,4,5,6]}
	\\

	I first used performed zero-crossing analysis using a LoG filter to create a gradient image. Zero-crossing image can be seen in the first swatch on the left in Figure \ref{fig:p3a}. I then performed this hierarchical reconstruction. This time, all patches, with the exception of, the first were down-sampled by using the zero-crossing "map" as a mask. Results at each level can be seen in Figures \ref{fig:p3a}, \ref{fig:p3b} and \ref{fig:p3c}. PSNRs can be found in the plot titles of each reconstruction swatch.
\end{homeworkProblem}






\begin{figure}[h!]
	\centering
	\includegraphics[width=\linewidth]{/home/komelmerchant/Desktop/JHUCourseTracking/ImageAndVideoProcessing/final/latex/p1.png}
	\caption{Reconstruction results and PSNRs for a, b, and c.}
	\label{fig:p1}
\end{figure}




\begin{figure}[h!]
	\centering
	\includegraphics[width=\linewidth]{/home/komelmerchant/Desktop/JHUCourseTracking/ImageAndVideoProcessing/final/latex/p2.png}
	\caption{Entropy Bar Graph for each sub-band}
	\label{fig:p2}
\end{figure}


\begin{figure}[h!]
	\centering
	\includegraphics[width=\linewidth]{/home/komelmerchant/Desktop/JHUCourseTracking/ImageAndVideoProcessing/final/latex/p3-a.png}
	\caption{LLLL Reconstruction From Edge-based Reconstruction}
	\label{fig:p3a}
\end{figure}


\begin{figure}[h!]
	\centering
	\includegraphics[width=\linewidth]{/home/komelmerchant/Desktop/JHUCourseTracking/ImageAndVideoProcessing/final/latex/p3-b.png}
	\caption{LL Reconstruction From Edge-based Reconstruction}
	\label{fig:p3b}
\end{figure}


\begin{figure}[h!]
	\centering
	\includegraphics[width=\linewidth]{/home/komelmerchant/Desktop/JHUCourseTracking/ImageAndVideoProcessing/final/latex/p3-c.png}
	\caption{Full Image Reconstruction From Edge-based Reconstruction}
	\label{fig:p3c}
\end{figure}










\pagebreak


\end{document}

